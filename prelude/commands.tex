% !TEX root = ../thesis.tex

% Create a 'python' environment that executes verbatim what is
% inside the environment as python code.
\usepackage{fancyvrb}
\newenvironment{python}
  {\VerbatimOut{runpython.tmp}}
  {\endVerbatimOut\immediate\write18{python runpython.tmp}}

% A macro that expands to the Git hash of the current branch.
% Upon compilation we write it to `version.tmp` (if shell escape is enabled),
% then we create the macro \gitversion that expands to the contents of this file.
% Put this info in the pdf metadata, along with title and author.
\makeatletter
\immediate\write18{git rev-parse HEAD > version.tmp}
\newread\@gitversionread
\openin\@gitversionread = version.tmp
\read\@gitversionread to \gitversion
\newcommand*{\makepdfmetadata}{
  \hypersetup{pdfinfo = { Title = \@title, Author = \@author, Version = \gitversion}}
}
\makeatother

% Add counters for theorems and definitions (for anything referenced, basically).
% Make it reset per chapter.
\newcounter{theorem}[chapter]
\renewcommand*{\thetheorem}{\thechapter.\arabic{theorem}}

% Have a generic command for theorem, lemma, etc.
% Optionally with a label, and after that an optional name.
\DeclareDocumentCommand{\theoremlike}{m g o}{\refstepcounter{theorem}\IfNoValueF{#3}{\label{#3}}\textbbf{#1 \thetheorem\IfNoValueF{#2}{ ∙ #2}} ∙\xspace}

\newcommand*{\lemma}{\theoremlike{Lemma}}
\newcommand*{\proposition}{\theoremlike{Proposition}}
\newcommand*{\theorem}{\theoremlike{Theorem}}
\newcommand*{\corollary}{\theoremlike{Corollary}}
\newcommand*{\definition}{\theoremlike{Definition}}
\newcommand*{\example}{\theoremlike{Example}}
\newcommand*{\notation}{\theoremlike{Notation}}
\newcommand*{\proof}{\emph{Proof}:\xspace}

% Define an equation environment `equationref` that numbers equations with the same counter as theorems.
\newenvironment*{equationref}{\refstepcounter{theorem}\renewcommand*{\theequation}{\thetheorem}\equation}{\endequation}

% Named sets and operators
\DeclareMathOperator{\id}{id}
\DeclareMathOperator{\im}{im}
\DeclareMathOperator{\Aut}{Aut}
\DeclareMathOperator{\GL}{GL}
\DeclareMathOperator{\Homeo}{Homeo}
\DeclareMathOperator{\Mat}{Mat}
\DeclareMathOperator{\SO}{SO}
\DeclareMathOperator{\Span}{Span}
\DeclareMathOperator{\SU}{SU}
\DeclareMathOperator{\Supp}{Supp}
\DeclareMathOperator{\Tr}{Tr}

% Sets
\providecommand*{\C}{}
\providecommand*{\H}{}
\providecommand*{\P}{}
\renewcommand*{\C}{\mathbb{C}}
\renewcommand*{\H}{\mathbb{H}}
\renewcommand*{\L}[1]{{\textstyle\bigwedge\nolimits^{\!{#1}}}}
\renewcommand*{\M}{\mathcal{M}}
\renewcommand*{\P}{\mathbb{P}}
\newcommand*{\CZ}{\C^2_\circ}
\newcommand*{\Cinf}{\mathcal{C}^{\hspace{1pt}\infty}}
\newcommand*{\GLC}{\GL_2(\C)}
\newcommand*{\GLR}{\GL_4(\R)}
\newcommand*{\Lk}{\L{k}}
\newcommand*{\Ll}{\L{l}}
\newcommand*{\Ln}{\L{n}}
\newcommand*{\N}{\mathbb{N}}
\newcommand*{\opp}{\textup{opp}}
\newcommand*{\PC}{\P^{\hspace{0.8pt}1}\hspace{-0.3pt}(\hspace{-0.7pt}\C\hspace{-0.3pt})}
\newcommand*{\R}{\mathbb{R}}
\newcommand*{\Rpos}{\R_{>0}}
\newcommand*{\SC}{S^3_\C}
\newcommand*{\SOR}{\SO_3(\R)}
\newcommand*{\SUC}{\SU_2} % The C is implicit in unitary.
\newcommand*{\Tp}{T_{\!p}}
\newcommand*{\Tps}{T^*_{\hspace{-0.1em}p}\!}
\newcommand*{\Tpss}{T^{**}_{\hspace{-0.1em}p}\!}
\newcommand*{\Ts}{T^*\!}
\newcommand*{\Z}{\mathbb{Z}}

% Arrows
\newcommand*{\inj}{\hookrightarrow}
\newcommand*{\longmapsfrom}{\leftarrow\!\rightfootline}
\newcommand*{\longto}{\longrightarrow}
\newcommand*{\surj}{\twoheadrightarrow}

% Boldface symbols, with ‘f’ suffix (for ‘field’)
\newcommand*{\Af}{\mathbf{A}}
\newcommand*{\Bf}{\mathbf{B}}
\newcommand*{\Ef}{\mathbf{E}}
\newcommand*{\jf}{\mathbf{j}}
\newcommand*{\nf}{\mathbf{\hat{n}}}
\newcommand*{\rf}{\mathbf{r}}
\newcommand*{\vf}{\mathbf{v}}

% Miscellaneous shorthands
\newcommand*{\set}[1]{\{ #1 \}}
\newcommand*{\inp}[2]{\langle #1, #2 \rangle}
\newcommand*{\nsq}[1]{\|#1\|^2}
\newcommand*{\eqdef}{\overset{\textup{def}}{=}}
\newcommand*{\qed}{\hfill\ensuremath{\square}}
\newcommand*{\mhd}{\textsc{\addfontfeature{LetterSpace = 5}mhd}\xspace}
\newcommand*{\modsim}{{/\!\sim}}
\newcommand*{\ppart}{\partial/\partial}
\newcommand*{\df}{d\!f} % Kern manually

% Lengths used in various pictures
\newlength{\kr}
