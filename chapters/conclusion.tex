% !TEX root = ../thesis.tex

\chapter{Conclusion}
In this thesis we have given two equivalent definitions of the Hopf map,
and with these we parametrised its fibres and
showed that they are all linked with one another.
We have given a procedure for constructing a divergenceless vector field
from a differentiable function from $\R^3$ to a two-dimensional manifold,
and applied this to the Hopf map composed with stereographic projection.
We explored how variations of the field can be constructed
by pulling back different two-forms
or by altering the differentiable function $\R^3 \to S^2\!$.
Finally, we interpreted the divergenceless vector field
obtained from the Hopf map as the magnetic field in \mhd,
and we gave a heuristic argument as to why this field exhibits a form of self-stability.
Areas of future research could be quantifying the degree of stability
and exploring extensions of the given procedure to electromagnetism.
