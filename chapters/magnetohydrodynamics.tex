% !TEX root = ../thesis.tex

\chapter{Magnetohydrodynamics}
\label{chap:magnetohydrodynamics}
Magnetohydrodynamics (henceforth abbreviated \mhd\!\!)
is a combination of the theories of fluid dynamics,
governed by the Navier-Stokes equations,
and the theory of electrodynamics,
governed by Maxwell’s equations.
The field of \mhd studies electrically conducting fluids,
the prime example of which are plasmas.

Plasma physics has promising applications such as nuclear fusion,
a source of energy that unlike nuclear fission does not produce radioactive byproducts.
A big problem here is the issue of \emph{plasma confinement}:
for controlled fusion, the plasma must be contained in a reactor vessel.
However, the temperature required for nuclear fusion is so high (above 150 million \textsc{°c}),
that no known material is able to withstand this amount of heat.
The plasma will melt the walls of the reactor if it makes contact with them.
To avoid this, present-day reactors employ intense magnetic fields to confine the plasma.

Alternatives have been proposed,
where the plasma has an inherent stability due to the structure of its magnetic field.
In order to understand this,
we will first give a bit of background about \mhd.
As we will see, the magnetic field is a key ingredient of this theory,
but the electric field plays a secondary role at best.
Next, we will show how knots and links can be used to provide stability.
We formalise this concept with the idea of \emph{helicity}.
Finally, we construct several magnetic fields with high helicity.

\section{Ideal magnetohydrodynamics}
\label{sec:ideal-mhd}
Ideal \mhd describes the dynamics of a conducting fluid with no net charge.
The quantities that play a role here are:
\begin{itemize}
\setlength{\itemsep}{-0.5em}
\item The mass density $\rho$
\item The fluid velocity $\vf$ \pagebreak
\item The pressure $p$
\item The magnetic field $\Bf$
\end{itemize}
Vector quantities have been set in boldface.
Note that there is no electric field here;
in \mhd the electric field is fully determined by \kern0.1pt $\vf$ and $\Bf$.
In ideal \mhd, the electric resistivity of the fluid is assumed to be zero.
That is, the fluid is a perfect conductor.
This gives us a first hint about why the electric field may be neglected:
in electrostatics, the electric field inside a perfect conductor is zero.
However, \mhd is not a static theory.
The reason that the electric field is secondary nonetheless,
is that the magnitude of the electric field is of the order $|\vf||\Bf|$,
as will be shown below.
Whereas the time derivative of the electric field does play a role in electromagnetic waves,
where it has a magnitude of the order $c\,|\Bf|$,
its contribution is negligible in \mhd when the velocity $\vf$ is nonrelativistic.

The evolution of a system in ideal \mhd,
ignoring the effects of gravity,
is given by the following equations (see also \parencite[p.~133]{goedbloed2004}):
\begingroup
\addtolength{\jot}{1em}
\begin{align}
\label{eqn:continuity-equation} \frac{\partial \rho}{\partial t} &= -\nabla \cdot (\rho \vf) \\
\label{eqn:cauchy-momentum}     \rho \left( \frac{\partial}{\partial t} + \vf \cdot \nabla \right) \vf &= \frac{1}{\mu_0}(\nabla \times \Bf) \times \Bf -\nabla p \\
\label{eqn:internal-energy}     \frac{\partial p}{\partial t} &= - \vf \cdot \nabla p - \gamma p \nabla \cdot \vf \\
\label{eqn:faraday}             \frac{\partial \Bf}{\partial t} &= \nabla \times {(\vf \times \Bf)} \\
\label{eqn:no-monopoles}        \nabla \cdot \Bf &= 0
\end{align}
\endgroup
Equation~\ref{eqn:continuity-equation}, the \emph{continuity equation} embodies conservation of mass:
if the mass density changes, the fluid must have flowed somewhere else.
Equation~\ref{eqn:cauchy-momentum}, the \emph{momentum equation}, describes the forces acting on the fluid.
The left-hand side represents the change in momentum,
the right hand side has a Lorentz force term (\hspace{1pt}$\jf \times \Bf$,
where $\jf = \mu_0^{-1} \nabla \times \Bf$ with $\mu_0$ the magnetic permeability of the vacuum)
and a pressure term.
Equation~\ref{eqn:internal-energy} concerns the \emph{internal energy} of the fluid.
A flow in the direction of the pressure gradient will reinforce the gradient (the first term),
and if there is a net influx of fluid into a volume,
pressure will build up in this volume (the second term).
Here the constant $\gamma$ is the \emph{adiabatic index},
the ratio of the heat capacity at constant pressure and the heat capacity at constant volume.
Equation~\ref{eqn:faraday} represents Faraday’s law, $\partial \Bf / \partial t = - \nabla \times \Ef$.
The expression $\Ef = - \vf \times \Bf$ is Ohm’s law for a perfect conductor.
Because of the infinite conductance of the fluid,
any electric field will vanish in the reference frame of a test particle moving with the fluid.
In the lab frame, we then find an electric field $-\vf \times \Bf$.
Finally, the magnetic field has no charges (equation~\ref{eqn:no-monopoles}).

\section{Linked and knotted fields}
\label{sec:linked-and-knotted-fields}
Linking and knotting provide a promising way of creating stable plasmas,
because in ideal \mhd they are preserved.
This imposes constraints on the evolution of the system.
Consequently, plasmas with linked or knotted field lines
might not be able to relax to a state of global minimum energy.
In this manner, topological properties of the field can provide stability.

In ideal \mhd, field lines of the magnetic field are said to be \emph{frozen in} in the fluid.
This idea, which is encoded in equation~\ref{eqn:faraday},
was hinted at in \parencite{alfven1942} and is sometimes called \emph{Alfvén’s theorem}.
It states that the magnetic flux through a surface does not change
as the surface moves along with the fluid flow.
From this principle it can be derived that points connected by a magnetic field line
will remain connected by the same field line as they move with the fluid.
In particular, field lines cannot pass through one another.
Linked field lines will stay linked throughout the evolution of the system.
A thorough derivation of these effects can be found in \parencite{stern1966}.

To study the dynamics of a conducting fluid,
consider a circular flux tube (a surface of magnetic field lines)
with high flux inside the tube, and zero flux outside.
We assume that $V \approx 2\pi r \itTheta$
is a good approximation of the volume $V$ of this tube,
where $2\pi r$ is the length of the tube,
and $\itTheta$ the surface area of its cross section.
Recall that the magnetic energy is given by
\[ E_B = \tfrac{1}{2} \iiint \!\!|\Bf|^2 \, dx^3 \]
If we assume a magnetic field of constant magnitude $B$ inside the tube,
then the flux $\itPhi$ through a cross section perpendicular to the field
is simply $B\itTheta$,
and conversely $B = \itPhi / \itTheta$.
We find that
\vspace{-0.5\parskip}
\[ E_B \approx \pi r \itTheta B^2 = \pi r \itTheta^{-1} \itPhi^2 \vspace{0.2\parskip} \]
By the frozen-in principle,
the flux $\itPhi$ through a cross section of the tube is constant in time.
Scaling the area $\itTheta$ by a factor $a$
while keeping $\itPhi$ constant,
will change the energy by a factor $a^{-1}$.
Thickening the tube will therefore decrease its magnetic energy.
On the other hand, scaling the length of the tube by a factor $b$
while keeping $\itPhi$ constant
will change the energy by a factor $b$.
Contracting the tube will decrease its magnetic energy.
From this we can conclude that an unconstrained flux tube
will contract and thicken as it relaxes.

The single flux tube helps us understand why linking is important for stability.
\marginfigure{
\tikzexternalenable
\tikzsetnextfilename{collapsing-flux-tubes}
\begin{center}
\hspace{-1em}
\begin{tikzpicture}
\newcommand*{\trefoil}[4]{
\begin{scope}
\tikzstyle{back} = [line width = #4, black]
\tikzstyle{knot} = [line width = #3, white]
\setlength{\kr}{#2}
\coordinate (A) at ($ #1 + ( 90 : 0.7\kr)$);
\coordinate (B) at ($ #1 + (210 : 0.7\kr)$);
\coordinate (C) at ($ #1 + (330 : 0.7\kr)$);
\coordinate (Z) at ($(B) !.5! (C) + (90 : \kr)$);
\draw[back] ($(C) + (330 : \kr)$) arc (-30 :  90 : \kr) -- (Z);
\draw[knot] ($(C) + (330 : \kr)$) arc (-30 :  90 : \kr) -- (Z);
\draw[back] ($(B) + (210 : \kr)$) arc (210 : 330 : \kr) --
            ($(A) + (330 : \kr)$) arc (-30 :  90 : \kr);
\draw[knot] ($(B) + (210 : \kr)$) arc (210 : 330 : \kr) --
            ($(A) + (330 : \kr)$) arc (-30 :  90 : \kr);
\draw[back] ($(A) + ( 88 : \kr)$) arc ( 88 : 210 : \kr) --
            ($(C) + (210 : \kr)$) arc (210 : 332 : \kr);
\draw[knot] ($(A) + ( 86 : \kr)$) arc ( 86 : 210 : \kr) --
            ($(C) + (210 : \kr)$) arc (210 : 334 : \kr);
\draw[back] (Z) + (0.1pt, 0) -- ($(B) + ( 90 : \kr)$) arc ( 90 : 212 : \kr);
\draw[knot] (Z) + (0.5pt, 0) -- ($(B) + ( 90 : \kr)$) arc ( 90 : 214 : \kr);
\end{scope}}
\trefoil{(0, 7.4em)}{2.0em}{3.0pt}{3.94pt}
\trefoil{(0,   0em)}{1.5em}{4.5pt}{5.44pt}
\trefoil{(0,-6.0em)}{1.0em}{10pt}{10.94pt}
\end{tikzpicture}
\end{center}
\vspace{1em}
\caption{Like linked flux tubes, a knotted flux tube cannot collapse onto itself.}}
Suppose that instead of a single flux tube,
we have two linked tubes.
Because the field lines cannot pass through eachother,
linking prevents the tubes from collapsing onto theirselves.
In order for one tube to contract,
the other tube must become thinner —
increasing its flux density and thereby its magnetic energy.
This is the way in which linking provides stability.
A more rigorous discussion can be found in \parencite{moffatt1969} and \parencite{arnold1974}.
The degree of stability that different types of links and knots provide
is an area of active research,
but this is beyond the scope of this thesis.

To formalise the concept of linked field lines,
we introduce a new quantity.

\definition
The \emph{magnetic helicity} of the magnetic field $\Bf = \nabla \times \Af$
in a bounded volume $V\!$,\, such that $\Bf \cdot \nf = 0$ on the boundary of $V$
where $\nf$ is a unit length normal vector of the boundary,
is defined by
\[ H_V \, = \!\iiint_{\!V} \Af \cdot \Bf \ dx^3 \]
An example of such a volume $V$ is a \emph{flux tube},
a volume formed by all of the field lines that intersect a certain surface.
See also \parencite[p.~156]{goedbloed2004}.

The definition above is a special case of the Hopf invariant
as defined in definition~\ref{def:hopf-invariant-of-subset}.
In section~\ref{sec:the-hopf-invariant} we proved the gauge invariance of $H_V$,
a property that is not obvious from its definition.
Furthermore, we showed that $H_V$ is invariant when
both $V$ and the field lines of $\Bf$ are transformed
by an orientation-preserving function.
The flow of a fluid is an example of such a function,
and because of the frozen-in principle,
the magnetic field does move along with the fluid flow.
It follows that in ideal \mhd
\[ \frac{dH_V}{dt} = 0 \]
The conservation of this quantity was discovered by \parencite{woltjer1958}.
An alternative derivation can be found in \parencite[p.~157]{goedbloed2004}.
Besides the helicity in $V$
it is possible to define a global helicity by integrating over all space,
but this requires restrictions on the field
if the helicity is to be gauge invariant.

To show the relation between linking and helicity,
consider again the circular flux tube for which
$V \approx 2 \pi r \itTheta$ is a good approximation of its volume.
Here $\itTheta$ is the surface area of its cross section.
Suppose that $\Bf$ inside the tube has constant magnitude $B$,
and zero magnitude outside of the tube.
Assume that this flux tube is linked once with an identical flux tube,
rotated by 90 degrees with respect to the first one.
We will denote the first tube by $T_1$ and the second tube by $T_2$.
To compute the magnetic helicity inside $T_1$,
we can factor the integral into a part along the field,
and a part perpendicular to the field.
Denote by $\sigma$ a field line inside $T_1$,
and let $D$ be the disk of which $\sigma$ is the boundary.
Let $dl$ be a line element along $\sigma$,
and $dS$ a surface element of $D$.
Note that $dl$ is parallel to $\Bf$.
We find
\[ H_{T_1}
 = \iiint_{\!T_1} \! \Af \cdot \Bf \ dx^3
\ \approx \ \itTheta B \oint_\sigma \! \Af \cdot dl
 = \itTheta B \iint_{\!D} (\nabla \times \Af) \cdot dS
 = \itTheta B \iint_{\!D} \! \Bf \cdot dS
\ \approx \ (\itTheta B)^2 \]
Here we used Stokes’ theorem to write the integral
as an integral of $\Bf$ over $D$,
which picks up a factor $\itTheta B$ from $T_2$ passing through it once.
The field inside $T_1$ does not contribute,
because $\Bf$ is perpendicular to the surface normal of $D$ there.
Beware that although we are computing the helicity \emph{inside} $T_1$,
it depends on $\Af$ and $\Bf$ \emph{outside} of $T_1$.
If $T_2$ would be wound around $T_1$ twice instead of once,
we would get an extra factor $2$.
In general,
when $T_2$ and $T_1$ are linked $n$ times,
the helicity in $T_1$ (and by symmetry, in $T_2$)
will be given by $n (\!\itTheta B)^2$.
The factor $n$ is how topology enters into \mhd.

When resistivity of the fluid is incorporated (non-ideal, resistive, or dissipative \mhd),
the frozen-in principle no longer holds.
Among others, an extra term must be added to equation~\ref{eqn:faraday}.
\parencite[p.~162]{goedbloed2004}
It follows that magnetic flux is no longer conserved,
and field lines may break and recombine.
The tools of topology break down here:
continuity is at the heart of topology,
so if field lines can break,
we cannot meaningfully speak about linking.
Fortunately, for suitable boundary conditions the helicity remains a well-defined quantity,
and as argued in \parencite{taylor1974},
helicity is \emph{approximately conserved},
meaning that it changes at timescales much larger
than typical timescales of fluid dynamics.
The extent to which helicity still provides stability in resistive \mhd
is beyond the scope of this thesis,
but work is being done in this area.

\section{Constructing a magnetic field}
\label{sec:constructing-a-magnetic-field}
In the previous sections we showed that we can construct
self-stable plasma configurations in \mhd
by giving a vector field with high helicity.
The field derived in section~\ref{sec:constructing-a-vector-field} comes to mind:
its field lines are the fibres of the Hopf map
projected stereographically onto $\R^3\!$,
so all of the field lines are linked with every other field line.

The approach taken in section~\ref{sec:constructing-a-magnetic-field}
was also used in \parencite{kamchatnov1982} to construct a magnetic field.
Kamchatnov only considered the pullback of $\omega_0$,
not of a general two-form.
In his case the vector potential was found in a deus ex machina manner,
but obviously this approach does not generalise to different two-forms on $S^2\!$.
By \poincares lemma the vector potential always exists,
but an explicit computation can get quite involved.
It was shown by Kamchatnov that the configuration obtained from the Hopf map
is a magnetohydrodynamic \emph{soliton}
— a wave that preserves its shape while propagating.

\begin{python}
from hopf import *
from sympy import exp

def identity(x):
    return x

def volume_form(x):
    return x

def south_form(x):
    d_sqr = sum(xi ** 2 for xi in sub(x, [-1, 0, 0]))
    f = exp(-d_sqr * 3)
    return [xi * f for xi in x]

def north_form(x):
    d_sqr = sum(xi ** 2 for xi in sub(x, [ 1, 0, 0]))
    f = exp(-d_sqr * 3)
    return [xi * f for xi in x]

window = [(x * 2.0 - 1.0) * 1.6 for x in interval_closed(79)]
write_field_energy('generated/energy-volume', volume_form, identity, window)
write_field_energy('generated/energy-south', south_form, identity, window)
write_field_energy('generated/energy-north', north_form, identity, window)
\end{python}
\begin{figure}[b!]
\margincaption[2.7em]{\label{fig:energy-density}
Magnetic energy density $\nsq{\Bf}$ in the planes $x_1 = 0$,
$x_2 = 0$ and $x_3 = 0$
for the pullback by $h \circ \pi^{-1}$ of the following functions on $S^2$:
$f(x) = 1$ for the top row,
$f(x) = \exp(-3\nsq{x + i})$ for the middle row, and
$f(x) = \exp(-3\nsq{x - i})$ for the bottom row.
Intensity has been normalised per row.}
\begin{center}
\tikzexternalenable
\tikzsetnextfilename{energy}
\begin{tikzpicture}
% Set custom heat-like colourmap that does not go all the way to black.
\pgfplotsset{
  colormap = {energy}{[1cm]
    rgb255(0cm) = (128, 0, 0);
    rgb255(3cm) = (255, 32, 0);
    rgb255(6cm) = (255, 192, 0);
    rgb255(8cm) = (255, 255, 255)
  },
}
\newcommand*{\energyplot}[1]{
  \nextgroupplot[xlabel = $x_2$, ylabel = $x_3$]
  \addplot3[surf, shader = interp] table {#1-x2x3.dat};

  \nextgroupplot[xlabel = $x_3$, ylabel = $x_1$]
  \addplot3[surf, shader = interp] table {#1-x3x1.dat};

  \nextgroupplot[xlabel = $x_1$, ylabel = $x_2$, colorbar]
  \addplot3[surf, shader = interp] table {#1-x1x2.dat};
}
\begin{groupplot}[group style = {
                    group size = 3 by 3,
                    horizontal sep = 3.5em,
                    vertical sep = 3.5em
                  },
                  width = 0.35\textwidth, height = 0.35\textwidth,
                  view = {0}{90},
                  point meta min = 0,
                  point meta max = 1,
                  ylabel style = {rotate = -90},
                  ylabel shift = -0.5em,
                  colorbar style = {
                    ytick = {0.0, 0.5, 1.0},
                    width = 1em
                  },
                  tick style = {
                    white,
                    major tick length = 0.3em
                  }]
  \energyplot{generated/energy-volume}
  \energyplot{generated/energy-south}
  \energyplot{generated/energy-north}
\end{groupplot}
\end{tikzpicture}
\end{center}
\end{figure}

By the principle given at the end of section~\ref{sec:constructing-a-vector-field},
there are two ways to generalise the field given in equation~\ref{eqn:hopf-field}
which we now take to be the magnetic field.
Firstly, we can pull back a different two-form on the sphere.
As this affects the magnitude of the field but not its field lines,
this allows us to control the \emph{energy density} of the magnetic energy of the field.
(See figure~\ref{fig:energy-density}.)
Secondly,
we may pull back by a different function, or even from a different manifold altogether.
The convenient property of linked field lines is a consequence of the Hopf map,
so this we do not change.
Instead, we will intersperse a differentiable function $g : S^3 \to S^3\!$,
and pull back by the composition
\begin{center}
\begin{tikzcd}
\R^3 \ar[r, hook, "\pi^{-1}"] &
S^3  \ar[r, "g"] &
S^3  \ar[r, two heads, "h"] &
S^2  \ar[r, hook, "i"] &
\R^3
\end{tikzcd}
\end{center}
Many functions $g$ could potentially be interesting here
and perhaps future research can be done in this area.
For instance, by considering $S^3$ as a subgroup of $\H$ as in section~\ref{sec:quaternions},
the map
\[ S^3 \longto S^3, \quad q \longmapsto q^n \]
is a differentiable function for all $n \in \Z$.
If we take $S^3 \subseteq \CZ$ instead,
the following map is interesting:
\begin{equationref}
\label{eqn:knot-map}
S^3 \longto S^3, \quad (z_1, \, z_2) \longmapsto \tau(z_1^n, \, z_2^m)
\end{equationref}
Here $m, n \in \Z$ and $\tau: \CZ \surj S^3$ denotes projection onto the sphere.
The above map is differentiable because it is the composition of $\tau$ with a polynomial.
For $m, n \notin \Z$ the map is not differentiable;
it is not even continuous,
so it does not make sense to compute the pullback by such a map.
It turns out that for coprime $m, n$ the field lines of the field induced by this map form torus knots.
In this way we can produce not only linked field lines,
but also knotted field lines.
See also figure~\ref{fig:knotted-field-lines}.
A slightly different map,
\[ S^3 \longto S^3, \quad (z_1, z_2) \longmapsto (z_1^{(n)},\, z_2^{(m)}) \]
can occasionally be found in literature.
Here the map $z \mapsto z^{(n)}$ denotes multiplying the argument of $z$ with $n$.
Unfortunately the map $z \mapsto z^{(n)}$ is not differentiable in $0$,
so the above function is not differentiable.
It has been used nevertheless in \parencite{arrayas2012},
albeit in a different construction.

More generally we could consider the map
\[ S^3 \longto S^3, \quad (z_1, \, z_2) \longmapsto \tau(p(z_1, z_2),\, q(z_1, z_2))  \]
where $p, q \in \C[Z_1, Z_2]$ are polynomials that have no common roots except for $(0, 0)$.
For polynomials with a common root other than $(0, 0)$ the function
would map some $(z_1, z_2) \in S^3$ to $(0, 0)$,
but this is not an element of $\CZ$;
there is no way to project the origin onto the three-sphere.

\begin{python}
from hopf import *
from math import pi
box       = [1.636, 1.636, 1.636]
pr        = orthographic_projection(pi * 0.12, pi * -0.1)
points    = [[[0.0, cos(pi * 0.0 / 15.0) * 1.30, sin(pi * 0.0 / 15.0) * 1.30], 'f1c'],
             [[0.0, cos(pi * 1.0 / 15.0) * 1.30, sin(pi * 1.0 / 15.0) * 1.30], 'f2c'],
             [[0.0, cos(pi * 5.0 / 15.0) * 1.62, sin(pi * 9.0 / 15.0) * 1.62], 'f3c']]
fibres    = [[compose(pr, projected_knot_fibre_through(x, 2, 3)),
              'front, ' + st, 'back'] for [x, st] in points]
cmds      = generate_raw_draw_2d(0.08, fibres)
box       = generate_raw_box_2d(box, pr, 'box')
write_items('generated/field-lines-trefoil.tikz', cons(box, cmds))

fibres    = [[compose(pr, projected_knot_fibre_through(x, 2, 5)),
              'front, ' + st, 'back'] for [x, st] in points]
cmds      = generate_raw_draw_2d(0.08, fibres)
write_items('generated/field-lines-cinquefoil.tikz', cons(box, cmds))
\end{python}
\begin{figure}
\margincaption[2em]{\label{fig:knotted-field-lines}
A few field lines of fields where $g \neq \id$;
the function from equation~\ref{eqn:knot-map} has been interspersed.
On the left, $m = 3$ and on the right $m = 5$.
In both cases $n = 2$.
The field lines form torus knots,
knotted themselves and linked with eachother.}
\tikzexternalenable
\tikzsetnextfilename{knotted-field-lines}
\hspace{0.5em} % No center here, manual alignment gets us better results.
\begin{tikzpicture}
\definecolor{f1c}{hsb}{0.6, 0.6, 0.5}
\definecolor{f2c}{hsb}{0.0, 0.8, 0.6}
\definecolor{f3c}{hsb}{0.5, 0.9, 0.6}
\tikzstyle{back}  = [line width = 3pt, white];
\tikzstyle{front} = [line width = 1pt];
\tikzstyle{box}   = [line width = 0.47pt];
\begin{scope}[scale = 1.1]
\input{generated/field-lines-trefoil.tikz}
\end{scope}
\begin{scope}[scale = 1.1, shift = {(5.5, 0)}]
\input{generated/field-lines-cinquefoil.tikz}
\end{scope}
\end{tikzpicture}
\vspace{1em}
\end{figure}

The construction used in this thesis to produce magnetic fields
is not limited to $\R^3$ or $S^2\!$,
and a generalisation of this procedure to electrodynamics could
potentially be interesting for future research.
Minkowski space $\M$ is a four-dimensional pseudo-Riemannian manifold,
where the bilinear form is given by the Lorentzian metric.
By mapping $\M$ to a two-dimensional manifold via a differentiable function,
we can construct a two-form $\omega \in \Omega^2 \M$ that satisfies $d\omega = 0$.
Maxwell’s source-free equations can be expressed neatly in the language of differential geometry
as
\[ d\xi = 0 \qquad\textup{and}\qquad d\hodge\xi = 0 \]
Here $\xi \in \Omega^2 \M$ can be identified with the electromagnetic field tensor
(sometimes called the Faraday tensor)
and $\hodge\xi$ denotes the \emph{Hodge dual} of $\xi$.
See \parencite[p.~502]{szekeres2004} for further information on
expressing Maxwell’s equations in this form.
With the construction in this thesis we can trivially satisfy
$d\xi = 0$, which corresponds to solving the two homogeneous equations
\[ \nabla \cdot \Bf = 0 \qquad\textup{and}\qquad \nabla \times \Ef + \frac{\partial \Bf}{\partial t} = 0 \]
The two-form $\xi$ will not automatically satisfy $d\hodge\xi = 0$ in general though.
It would be interesting to investigate whether functions $\M \to N$
exists for a two-dimensional manifold $N$
such that the pullback does satisfy $d\hodge\xi = 0$ trivially.
