% !TEX root = ../thesis.tex

\chapter{Introduction}
One of the problems in plasma physics is the issue of plasma confinement.
How does one confine a dense plasma to a reactor vessel for a sustained period of time?
Plasmas are extremely hot,
so any contact with the walls of a reactor would be fatal.
Solving this problem is an important step towards nuclear fusion,
a sustainable energy source that unlike nuclear fission does not produce radioactive byproducts.
Current efforts focus on repelling the plasma
from the walls of the reactor with intense magnetic fields,
although other options might be feasible.
One approach is that of self-stability,
where the magnetic field of the plasma prevents it from deforming too much.
In this thesis we will investigate how self-stability can arise,
and we will construct a few magnetic fields with desirable properties.

As we will see,
linking of the field lines is important for these magnetic fields.
This leads us to the Hopf map,
a differentiable function from $S^3$ to $S^2$ of which the fibres,
the inverse images of points on $S^2\!$,
are linked.
Before we can define the Hopf map,
we will recall some of the theory involved in chapter~\ref{chap:preliminaries},
and we will investigate a few useful group actions.
In chapter~\ref{chap:the-hopf-map} we will turn to the Hopf map itself.
Via stereographic projection we can visualise the fibres in $\R^3\!$,
and with ideas from topology we can quantify linking of the fibres.
To construct a vector field with field lines
based on the fibres of the Hopf map,
we use tools from differential geometry developed in chapter~\ref{chap:differential-forms}.
Finally we make the link to magnetohydrodynamics in chapter~\ref{chap:magnetohydrodynamics}.
The Hopf invariant,
a quantity that appears purely algebraic at first sight,
will turn out to have a direct physical interpretation as the helicity of a field,
a conserved quantity that plays a role in the stability of plasmas.

Chapter~\ref{chap:preliminaries} through~\ref{chap:differential-forms}
are mathematical in nature.
For physicists who are not familiar with the formalism,
or who care about results instead of proofs,
a paragraph with “physical interpretation” has been added after every section whenever possible.
When the theory does not admit a direct physical interpretation,
a paragraph “informal summary” has been added instead.
