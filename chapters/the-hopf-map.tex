% !TEX root = ../thesis.tex

\chapter{The Hopf map}
\label{chap:the-hopf-map}
\newcommand*{\hquat}{h_\textup{quat}}
\newcommand*{\hproj}{h_\textup{proj}}
The different definitions of $S^3$ and $S^2$ that were explored in the previous chapter
go along with different definitions of the Hopf map.
In this chapter we will give those definitions,
and show that they are equivalent in the following sense:
if $\hproj$ and $\hquat$ represent the projective and quaternionic definition of the Hopf map respectively,
then the following diagram commutes:

\vspace{-\parskip}
\begin{center}
\begin{tikzcd}
\SC \ar[r, two heads, "\hproj"]
    \ar[d, leftrightarrow] &
\PC \ar[d, leftrightarrow] \\
S^3 \ar[r, two heads, "\hquat"] &
S^2
\end{tikzcd}
\end{center}

$S^3$ and $S^3_\C$ were shown to be homeomorphic in proposition~\ref{prop:s3-equivalence}.
In theorem~\ref{thm:s2-homeom-p1c} it was stated that $\PC$ and $S^2$ are homeomorphic,
which we will be able to prove at last.
Finally, the group actions explored in the previous chapter will be used to examine the fibres of the Hopf map.

\section{The projective Hopf map}
\label{sec:hopf-projective}
As we saw in section \ref{sec:projective-space},
$\PC$ can be defined as a quotient of $\CZ$ with respect to the action of $\C^*\!$.
Because $\C^* \cong \Rpos \times S^1\!$,
the quotient map factors over $\SC$ by theorem~\ref{thm:quotient-map-factors},
where $\SC = \CZ \, / \, \Rpos$ as in definition \ref{def:s3-complex}.
This allows us to define the Hopf map:

\definition[def:hopf-projective]
The \emph{Hopf map} is the unique continuous map $h : \SC \surj \PC$
that makes the following diagram commute:
\begin{center}
\begin{tikzcd}[column sep = small] &
\CZ \ar[dl, "r", two heads, swap]
    \ar[dr, "q", two heads] & \\
\SC \ar[rr, "h", two heads] & &
\PC
\end{tikzcd}
\end{center}
By theorem~\ref{thm:quotient-map-factors},
the Hopf map is the quotient map of the $S^1$-action on $\SC$.

This definition tells a lot about the Hopf map already.
It shows that its \emph{fibres} — the inverse images of points in $\PC$ —
are orbits of the $S^1$-action;
the fibres can be parametrised by $S^1\!$.
In section~\ref{sec:fibres} we will explore the geometry of the fibres,
which will turn out to be great circles on $S^3\!$.
Furthermore,
because $h$ is the quotient map of a group action,
it is surjective, continuous, and open by proposition~\ref{prop:open-quotient-map}.

\section{The quaternionic Hopf map}
\label{sec:hopf-quaternionic}
In section \ref{sec:quaternions} we defined $S^3$ and $S^2$ as subsets of $\H$,
with $S^3$ acting on $S^2\!$.
With this action,
we can define the Hopf map as follows:

\definition[def:hopf-quaternionic]
The \emph{Hopf map} is the map
\[ h : S^3 \longto S^2,
   \quad q \longmapsto q^{-1} \cdot i \]
Recall that ‘${}\cdot{}$’ denotes the action,
$q^{-1} \cdot i = q^{-1} i q$.
For quaternion multiplication we will simply use juxtaposition.
The reason that we choose $q^{-1} \cdot i$ here instead of $q \cdot i$,
will become clear in theorem \ref{thm:hopf-map-equivalence}.
An other way to think of this,
is that $h$ is the map $q \mapsto i \cdot q = q^{-1} i q$,
where $\H^*$ acts \emph{from the right} on $\H$.
The quaternionic definition is more suitable for doing computations
than the projective definition,
because it allows us to work with Cartesian coordinates on $S^2\!$.
The image of the quaternion $q = a + bi + cj + dk \in S^3$
under the Hopf map is given by
\begin{equationref}
\label{eqn:hopf-coordinates}
\begin{aligned}
h(q) &= q^{-1} \cdot i = q^{-1} \, i \, q = \overline{q} \, i \, q
  \\ &= (a - bi - cj - dk) \, i \, (a + bi + cj + dk)
% \\ &= (a - bi - cj - dk) (ai - b + ck - dj)
% \\ &= a^2 i - ab + ack - adj + (- bi - cj - dk) (ai - b + ck - dj)
% \\ &= a^2 i - ab + ack - adj + ab + b^2i + bcj + bdk + (- cj - dk) (ai - b + ck - dj)
% \\ &= a^2 i - ab + ack - adj + ab + b^2i + bcj + bdk + ack + bcj - c^2i - cd + (- dk) (ai - b + ck - dj)
% \\ &= a^2 i - ab + ack - adj + ab + b^2i + bcj + bdk + ack + bcj - c^2i - cd - adj + bdk + cd - d^2i
% \\ &= (a^2 + b^2 - c^2 - d^2) i - ab + ack - adj + ab + bcj + bdk + ack + bcj - cd - adj + bdk + cd
% \\ &= (a^2 + b^2 - c^2 - d^2) i + (ab - ab) + ack - adj + bcj + bdk + ack + bcj - cd - adj + bdk + cd
% \\ &= (a^2 + b^2 - c^2 - d^2) i + (ab - ab) + (cd - cd) + ack - adj + bcj + bdk + ack + bcj - adj + bdk
% \\ &= (a^2 + b^2 - c^2 - d^2) i + (ab - ab) + (cd - cd) + (ack + ack) - adj + bcj + bdk + bcj - adj + bdk
% \\ &= (a^2 + b^2 - c^2 - d^2) i + (ab - ab) + (cd - cd) + (ack + ack) + (bcj + bcj) - adj + bdk - adj + bdk
% \\ &= (a^2 + b^2 - c^2 - d^2) i + (ab - ab) + (cd - cd) + (ack + ack) + (bcj + bcj) + (bdk + bdk) - (adj + adj)
% \\ &= (a^2 + b^2 - c^2 - d^2) i + (ack + ack) + (bcj + bcj) + (bdk + bdk) - (adj + adj)
% \\ &= (a^2 + b^2 - c^2 - d^2) i + 2ack + 2bdk + (bcj + bcj) - (adj + adj)
% \\ &= (a^2 + b^2 - c^2 - d^2) i + 2ack + 2bdk + 2bcj - 2adj
  \\ &= (a^2 + b^2 - c^2 - d^2) i + 2 (bc - ad) j + 2 (ac + bd) k
\end{aligned}
\end{equationref}
Because $h$ is given by polynomial equations on every coordinate,
it is continuous.
Because $S^3$ acts transitively on $S^2$ by corollary~\ref{cor:transitive-s3-action},
$h$ is surjective.

The projective definition of the Hopf map in section \ref{sec:hopf-projective}
emphasises that the fibres of the Hopf map are orbits of a group action.
The quaternionic definition given here,
instead emphasises \emph{stabilisers} of a group action.
The fibre above $i$ consists of all $q \in S^3$ with $q^{-1} \cdot i = i$,
the stabiliser $S^3_i$ of $i$.
The fibre above $p \in S^2$ is a right coset of the stabiliser.
Because $S^3$ acts transitively on $S^2$ by corollary~\ref{cor:transitive-s3-action},
there exists an $x \in S^3$ such that $p = x \cdot i$.
The fibre above $p$ consists of all $q \in S^3$ such that $q^{-1} \cdot i = p$.
It follows that $q \cdot x \cdot i = q \cdot p = i$,
thus $qx$ stabilises $i$ and $h^{-1}(p) = S^3_i x^{-1}$.

At present, it is not at all obvious that the Hopf map as defined in
definition~\ref{def:hopf-projective} is related to the Hopf map as defined in definition~\ref{def:hopf-quaternionic}.
On the contrary: we have not even proven that the codomains $\PC$ and $S^2$ are homeomorphic.
Fortunately, we can prove both statements at once.

\theorem[thm:hopf-map-equivalence]
Denote by $\SC$ and $S^3$ the three-sphere as defined by
definition~\ref{def:s3-complex} and \ref{def:s3-real} respectively.
Let $\Psi : \CZ \to \H^*$ be the restriction
of the $\R$-linear isomorphism $(z_1, z_2) \mapsto z_1 + z_2 j$.
From proposition~\ref{prop:s3-equivalence} it follows that
$\Psi$ descends to $\psi : \SC \to S^3\!$.
Denote by $r$ the quotient map,
by $s$ projection onto $S^3\!$,
and by $\hproj$ and $\hquat$ the Hopf map as defined in
definition~\ref{def:hopf-projective} and \ref{def:hopf-quaternionic} respectively.
Then there exists a unique homeomorphism $\phi : \PC \to S^2$
that makes the following diagram commute:

\begin{center}
\begin{tikzcd}
\CZ  \ar[r, two heads, "r"]
     \ar[d, "\Psi"] &
\SC  \ar[r, two heads, "\hproj"]
     \ar[d, "\psi"] &
\PC  \ar[d, dashed, "\exists_! \phi"] \\
\H^* \ar[r, two heads, "s"] &
S^3  \ar[r, two heads, "\hquat"] &
S^2
\end{tikzcd}
\end{center}

\proof
We will show that $\Phi = \hquat \circ s \circ \Psi$ is compatible
with the quotient map $\CZ \surj \PC$.
Let $(z_1, z_2) \in \CZ$ and $z \in \C^*\!$,
such that $(z_1 : z_2) = (zz_1 : zz_2)$.
Because $\C^*$ stabilises $i \in S^2\!$,
we have
\begin{align*}
\Phi(z z_1, z z_2)
\, = \, (z z_1 + z z_2 j)^{-1} \cdot i
\, = \, (z_1 + z_2 j)^{-1} \cdot (z^{-1} \cdot i)
\, = \, (z_1 + z_2 j)^{-1} \cdot i
\, = \, \Phi(z_1, z_2)
\end{align*}
By the universal property of the quotient topology (theorem \ref{thm:universal-property-quotient-topology}),
there exists a unique continuous map $\phi$ that makes the diagram commute.

To give the inverse of $\phi$,
let a point $p \in S^2$ be given.
We saw before that there exists an $x \in S^3$ with $p = x \cdot i$,
such that $\hquat^{-1}(p) = S^3_i x^{-1}$.
Because $S^3_i = S^1 \subseteq \C^*\!$,
we can write every point in $S^3$ that maps to $p$ as $zx^{-1}$ for some $z \in S^1$,
and we can write $x^{-1}$ as $z_1 + z_2 j$ for some $z_1, z_2 \in \C$.
Therefore, all points in the fibre above $p$ map to $(z_1 : z_2)$
under $\hproj \circ \psi^{-1}$.
To show that this does not depend on the choice of $x$,
note that if we had $y \in S^3$ with $p = y \cdot i$,
then $x = yz^{-1}$ for some $z \in S^1\!$,
so $x^{-1} = z y^{-1}$.

Recall that $S^3_\C$ is compact by proposition \ref{prop:s3-equivalence},
so $\PC$ is compact,
for it is the continuous image of a compact space.
$S^2$ is Hausdorff because it is a subspace of $\H$,
which is Hausdorff.
Therefore, $\phi$ is a continuous bijection from a compact space to a Hausdorff space.
It follows that $\phi$ is a homeomorphism.
(See for instance theorem 3.3.11 of \parencite[p.~81]{runde2005}.)
\qed

Using equation~\ref{eqn:hopf-coordinates},
we can give an explicit expression for $\phi$.
Using equation~\ref{eqn:transitive-s3-action},
we can give an explicit expression for $\phi^{-1}$:
\begin{equationref}
\begin{aligned}
(a + bi : c + di) &\longmapsto (a^2 + b^2 - c^2 - d^2) i + 2 (bc - ad) j + 2 (ac + bd) k \\
(i + \alpha i : \beta + \gamma i) &\longmapsfrom \alpha i + \beta j + \gamma k
\end{aligned}
\end{equationref}
It is assumed here that $a^2 + b^2 + c^2 + d^2 = 1$
and $\alpha^2 + \beta^2 + \gamma^2 = 1$,
with $\alpha \neq -1$.
For $\alpha = -1$, we have $\phi^{-1}(-i) = (0 : 1)$.
Note that it is impossible to give a globally valid expression for $\phi^{-1}$:
the map $\phi \circ h : S^3_\C \surj S^2$ does not admit a global continuous section.
If it did, this would imply that $S^3 = S^2 \times S^1\!$,
which is not the case.

\subsection*{Informal summary}
In this section and in the previous section,
we have given two definitions of the Hopf map.
The projective definition as given in definition~\ref{def:hopf-projective}
can be written as $h : \SC \surj \PC$, $(z_1, z_2) \mapsto (z_1 : z_2)$.
Recall that if $z_2$ is nonzero, $(z_1 : z_2)$ may be thought of as $z_1 / z_2$.
This shows that multiplying $z_1$ and $z_2$ by $e^{it}$ for any $t \in \R$
does not change the image under the Hopf map.
It follows that the fibres of the Hopf map are circular,
a feature that will be explored further in the next section.
Definition~\ref{def:hopf-quaternionic} gives an alternative
definition of the Hopf map based on quaternions.
This definition is useful for doing computations,
because it allows us to work with Cartesian coordinates on $S^2\!$.
Theorem~\ref{thm:hopf-map-equivalence} shows that both definitions are equivalent.
This theorem also shows that $\PC$ and $S^2$
are \emph{homeomorphic},
meaning that for topological purposes
they are indistinguishable.

\section{Fibres}
\label{sec:fibres}
The Hopf map,
a surjective, continuous map from $S^3$ to $S^2\!$,
is interesting for many reasons.
The primary reason that we are interested in it here,
are its fibres.
Those are circles in $S^3$ that — as we will see in section~\ref{sec:linking} —
are linked, like keyrings can be linked.
Moreover, \emph{all} fibres are linked with \emph{every} other fibre.
Before we can study linking however,
we will first introduce the tools for studying the fibres.

In section~\ref{sec:hopf-quaternionic},
we saw already that the fibre above $p \in S^2$
is given by $S^1 x^{-1}$,
where $x \in S^3$ is such that $x \cdot i = p$.
In combination with equation~\ref{eqn:transitive-s3-action}
(an expression for $x$),
this allows us to explicitly parametrise the fibres of the Hopf map.
While such an expression is useful for computations,
it does not give us any geometrical insight.
Therefore, we will study the fibes of the Hopf map in a different way.
For this,
we will first revisit the $\GLC$ action on $\CZ$.

In section~\ref{sec:projective-space} we saw how $\GLC$ acts on $\CZ$.
Every element of $\GLC$ induces a homeomorphism $\CZ \to \CZ$.
These homeomorphisms are restrictions
of $\C$-linear (and thereby also $\R$-linear) automorphisms $\C^2 \to \C^2\!$,
which means the action descends to $S^3_\C$ and $\PC$.

\proposition[prop:gl2c-induces-actions]
Let $\alpha : \C^2 \to \C^2$ be a $\C$-linear automorphism.
Then there exist unique homeomorphisms $\beta, \gamma, \delta$
that make the following diagram commute:
\begin{center}
\begin{tikzcd}
\C^2   \ar[d, "\alpha"] &
\CZ    \ar[d, dashed, "\exists_! \, \beta"]
       \ar[l, hook, swap, "i"]
       \ar[r, two heads, "r"]
       \ar[rr, two heads, bend left, "q"{pos = 0.525}] &
S^3_\C \ar[d, dashed, "\exists_! \, \gamma"]
       \ar[r, two heads, "h"] &
\PC    \ar[d, dashed, "\exists_! \, \delta"] \\
\C^2   &
\CZ    \ar[l, hook, swap, "i"]
       \ar[r, two heads, "r"]
       \ar[rr, two heads, bend right, swap, "q"{pos = 0.525}] &
S^3_\C \ar[r, two heads, "h"] &
\PC
\end{tikzcd}
\end{center}
Here $q$ and $r$ denote the quotient maps,
$i$ denotes the inclusion,
and $h$ denotes the Hopf map.

\proof
The map $\beta$ is the restriction of $\alpha$ to $\CZ$.
The map $r \circ \beta$ is compatible with $r$
because $\beta$ is $\R$-linear;
if two elements are equivalent in $\CZ$,
then their images under $\beta$ are also equivalent.
By the universal property of the quotient topology
(theorem~\ref{thm:universal-property-quotient-topology}),
we get a unique continuous map $\gamma$.
By applying this argument to $\beta^{-1}$,
we find a unique continuous map $\gamma^{-1}$ which is the inverse of $\gamma$,
so $\gamma$ is a homeomorphism.
Similarly,
the map $q \circ \beta$ is compatible with $q$,
because if two elements in $\CZ$ differ by a factor $\lambda \in \C^*\!$,
then their images under $\beta$ differ by a factor $\lambda$,
as $\beta$ is $\C$-linear.
By the universal property of the quotient topology
we get a unique homeomorphism $\delta$ that makes the diagram commute.
\qed

This proposition tells us that the action of $\GLC$
descends naturally to $S^3_\C$ and $\PC$
by letting $g \in \GLC$ act on a representative.
Beware that although $\GLC$ acts on $\CZ$ by linear automorphisms,
the induced automorphisms are \emph{not} linear;
in general they are not linear automorphisms of $\R^4$ and $\R^3$
restricted to $S^3$ and $S^2\!$.

The general linear group $\GLR$ also acts on $\C^2\!$,
and by restriction on $\CZ$
when $\C^2$ is considered a four-dimensional real vector space.
This action induces an action of $\GLR$ on $S^3_\C$;
the same argument as before holds.
However, this action does \emph{not} induce an action on $\PC$,
because elements of $\GLR$ are not $\C$-linear automorphisms in general.

We saw already that the fibres of the projective hopf map are the
orbits of the $S^1$-action on $\CZ$.
This allows us to parametrise fibres easily.
For $(z_1 : z_2) \in \PC$,
if we assume that $|z_1|^2 + |z_2|^2 = 1$,
and if we work with $S^3$ instead of $S^3_\C$ by taking representatives of unit length,
we have:
\[ h^{-1}(z_1 : z_2) = \set{ (e^{it} z_1, e^{it} z_2) \in S^3 \mid t \in \R } \]
For the fibres above $(1 : 0)$ and $(0 : 1)$,
the geometrical picture is clear:
the fibres are unit circles in the planes spanned by $(1, 0)$ and $(i, 0)$,
and $(0, 1)$ and $(0, i)$ respectively.
Because the circles have unit radius,
they are great circles on $S^3\!$.
An other way to state this,
is that the fibres are precisely the intersections of $S^3$ with complex linear subspaces of dimension one
(in particular those are intersections of $S^3$ with real linear subspaces of dimension two).
For other points on $\PC$ however,
it is not immediately clear what the fibres look like.
This is where the $\GLC$-action is useful.
Via the homeomorphism $S^3_\C \to S^3$ from proposition~\ref{prop:s3-equivalence},
$\GLC$ acts on $S^3\!$.
If $g \in \GLC$ maps $(0 : 1)$ to $(z_1 : z_2)$,
then commutativity of the diagram in proposition~\ref{prop:gl2c-induces-actions}
means that $g$ maps the fibre above $(0 : 1)$ into the fibre above $(z_1 : z_2)$.
The fibre above $(0 : 1)$ is the intersection of a linear subspace
with $S^3\!$,
and because $g$ is linear,
the fibre above $(z_1 : z_2)$ is also the intersection of a linear subspace
with $S^3\!$.
Therefore it is a great circle as well.

\proposition[prop:glc-acts-transitively]
$\GLC$ acts transitively on $\PC$.

\proof
$\GLC$ acts transitively on $\CZ$.
Because $\PC$ is a quotient space of $\CZ$,
every element can be represented by an element of $\CZ$,
and because $\GLC$ acts transitively,
every representative can be reached from e.g. $(0, 1)$.
\qed

\corollary
All fibres of the Hopf map are great circles on $S^3\!$.
We saw already that for all $g \in \GLC$,
the fibre above $g \cdot (0 : 1)$ is a great circle,
and because $\GLC$ acts transitively,
every point in $\PC$ is of this form.

\section{Stereographic projection}
The fibres of the Hopf map that we investigated in the previous section
are subsets of $S^3\!$.
But $S^3$ can be hard to visualise;
it is a subset of a four-dimensional space.
Furthermore, with the goal of constructing a vector field on $\R^3$ in mind,
we somehow have to get to $\R^3\!$.
The way we can move between $\R^3$ and $S^3$ is by stereographic projection.
Given that $S^3$ and $\R^3$ are not homeomorphic,
we will have to make some concessions.
Fortunately $S^3$ is the one-point compactification of $\R^3\!$,
so if we want to go from $S^3$ to $\R^3$
we only loose a single point.
Nevertheless, this discrepancy will turn out to introduce some artefacts,
but these will in turn help to understand the geometry of $S^3\!$.

\definition
The \emph{stereographic projection} from $S^n$ onto $\R^n\!$,
with projection point $p \in S^n$ is given by
\[ \pi : S^n \setminus \set{p} \longto \R^n,
   \quad x \longmapsto \overline{px} \cap \R^n \]
Here we embed $\R^n$ in $\R^{n + 1}$ as $p^\perp\!$,
and $\overline{px}$ denotes the line connecting $p$ and $x$.

We will use coordinates $x_0, \ldots, x_n$ on $\R^{n + 1}$
and coordinates $x_1, \ldots, x_n$ on $\R^n\!$.
Setting $p = (1, 0, \ldots, 0)$
fixes the embedding of $\R^n$ in $\R^{n + 1}$ via $(x_1, \ldots, x_n) \mapsto (0, x_1, \ldots, x_n)$.
The connection line $\overline{px}$ can then be parametrised as $p + \lambda(x - p)$ for $\lambda \in \R$.
Intersecting this line with the hyperplane $x_0 = 0$ by setting
$p_0 + \lambda(x_0 - p_0) = 0$ yields $\lambda = - \frac{1}{x_0 - 1}$,
where the subscript zero denotes the first coordinate.
Substituting $\lambda$, we find the projection of $x$:
\begin{equationref}
\label{eqn:stereographic-projection}
   \pi(x_0, \ldots, x_n)
 = \left(-\frac{x_1}{x_0 - 1}, \ldots, -\frac{x_n}{x_0 - 1} \right)
 = \left(\frac{x_1}{1 - x_0}, \ldots, \frac{x_n}{1 - x_0} \right)
\end{equationref}
Conversely, if we have a point $x \in \R^n\!$,
then we can embed it in $\R^{n + 1}$ and parametrise the line through $x$ and $p$
as $p + \lambda(x - p)$ for $\lambda \in \R$.
This time we want to find the intersection with $S^n\!$,
so we must solve $\nsq{p + \lambda(x - p)} = 1$.
This yields $\lambda = \frac{2}{\nsq{x} + 1}$.
Substituting $\lambda$, we find the inverse image of $x$:
\begin{equationref}
\label{eqn:inverse-stereographic-projection}
    \pi^{-1}(x_1, \ldots, x_n)
% = \left(1 - \frac{2}{\nsq{x} + 1}, \frac{2x_1}{\nsq{x} + 1}, \ldots, \frac{2x_n}{\nsq{x} + 1} \right)
  = \frac{1}{\nsq{x} + 1} \left(\nsq{x} - 1, \, 2x_1, \, \ldots, \, 2x_n \right)
\end{equationref}
From equation~\ref{eqn:stereographic-projection} and \ref{eqn:inverse-stereographic-projection}
it is clear that $\pi$ and $\pi^{-1}$ are continuous.
It follows that $\pi$ is a homeomorphism between $S^n \setminus \set{p}$ and $\R^n\!$.

It turns out that the stereographic projection of a circle on $S^n$
is a circle in $\R^n\!$,
a property that will be useful when studying the fibres of $h \circ \pi^{-1}$.
To see why this is the case,
we will first study the more general mapping of spheres.

\definition
A \emph{sphere} in $S^n$ is the intersection of $S^n$
with a hyperplane given by $\inp{x}{\hat{n}} = t$,
where $\hat{n} \in S^n$ is a normal vector of the hyperplane,
and $t \in (-1, 1)$ is its offset to the origin.
We choose $|\,t\,| < 1$ such that the intersection is not empty or finite.

\example
A sphere in $S^2$ is simply a circle.
If $t = 0$, it is a great circle.

\definition
A \emph{sphere} in $\R^n$ with centre $x_c \in \R^n$ and radius $r \in \Rpos$
is the set
\[ \set{ x \in \R^n \mid r^{\,2} = \nsq{x - x_c} } \]
Note that a sphere in $S^n$ is the intersection of a sphere in $\R^{n + 1}\!$ with $S^n\!$.
By expanding the square,
we may alternatively write a sphere in $\R^n$
with centre $x_c$ and radius $r$ as
\begin{equationref}
\label{eqn:hypersphere}
\set{ x \in \R^n \mid r^{\,2} - \nsq{x_c} = \nsq{x} - 2 \inp{x}{x_c} }
\end{equationref}
Now we can turn to the relation between spheres in $S^n$ and in $\R^n\!$.

\proposition
Let $B \subseteq S^n$ be a sphere defined by the normal vector $\hat{n} \in S^n$
and offset $t \in (-1, 1)$.
Then for its image under the stereographic projection $\pi : S^n \setminus \set{p} \to \R^n\!$,
the following holds:
\begin{enumerate}
\item If $p \notin B$, $\pi(B)$ is a sphere in $\R^n\!$.
\item If $p \in B$, $\pi(B \setminus \set{p})$ is a hyperplane in $\R^n\!$.
\end{enumerate}

\proof
The image of $B$ or $B \setminus \set{p}$ when $p \in B$,
is given by the set of $x \in \R^n$ such that $\pi^{-1}(x) \in B$.
Embed $\R^n$ in $\R^{n + 1}$ as the hyperplane $x_0 = 0$.
Then we can write
\[ \pi(B)
\ = \ \set{x \in \R^n \mid \inp{\pi^{-1}(x)}{\hat{n}} = t }
\ = \ \set{x \in \R^n \mid n_0(\nsq{x} - 1) + \inp{x}{\hat{n}} = t } \]
Here $n_0$ denotes the first coordinate of $\hat{n}$.
If $n_0 \neq 0$,
we recognise equation~\ref{eqn:hypersphere},
so $\pi(B)$ is a sphere in $\R^n\!$.
If $n_0 = 0$,
then the predicate reduces to $\inp{x}{\hat{n}} = t$,
which is the equation for a hyperplane in $\R^n$ with normal $\hat{n}$
and offset $t$ to the origin.
Furthermore, $n_0 = 0$ if and only if $p \in B$.
\qed

\proposition
Let $n \geq 2$ and let $C \subseteq S^n$ be a circle,
the nonempty intersection of $n - 1$ spheres
defined by hyperplanes with linearly independent normal vectors.
Then for its image under the stereographic projection $\pi : S^n \setminus \set{p} \to \R^n\!$,
the following holds:
\begin{enumerate}
\item If $p \notin C$, $\pi(C)$ is a circle in $\R^n\!$.
\item If $p \in C$, $\pi(C \setminus \set{p})$ is a line in $\R^n\!$.
\end{enumerate}

\proof
We use the fact that for $U, V \subseteq S^n\!$,
we have $\pi(U \cap V) = \pi(U) \cap \pi(V)$.
As $C$ is the intersection of $n - 1$ spheres,
its image is the intersection of $n - 1$ spheres or hyperplanes.
These all lie in distinct hyperplanes with linearly independent normal vectors,
so the image is a subset of a two-dimensional plane in $\R^n\!$.
If $p \notin C$ then at least one of the images will be a sphere,
so $\pi(C)$ is a circle.
If $p \in C$ then all of the images will be distinct hyperplanes,
the intersection of which is a line.
\qed

\corollary[cor:circles-and-axis]
The fibres of $h \circ \pi^{-1}$ are all circles in $\R^3\!$,
except for the fibre above $(1 : 0)$ which is the $x_1$-axis.
Furthermore, from equation~\ref{eqn:stereographic-projection} it is clear
that the fibre above $(0 : 1)$ is the unit circle in the $x_2 x_3$-plane.
This has been visualised in figure~\ref{fig:two-fibres}.

\begin{figure}
\definecolor{f1c}{hsb}{0.6, 0.6, 0.5}
\definecolor{f2c}{hsb}{0.0, 0.8, 0.6}
\margincaption[3em]{\label{fig:two-fibres}
Fibres of the Hopf map visualised through stereographic projection;
the fibre above $ i \in S^2$ is the $x_1$-axis (coloured {\color{f2c}•}),
the fibre above $-i \in S^2$ is the unit circle in the plane $x_1 = 0$
(coloured {\color{f1c}•}).
See also figure~\ref{fig:more-fibres}.}
\begin{python}
from hopf import *
from math import pi
pr         = orthographic_projection(pi * 0.42, pi * -0.1)
fibre_x2x3 = [compose(pr, projected_fibre_from_spherical([-pi * 0.5, 0.0])), 'f1f', 'back']
fibre_x1_b = [compose(pr, projected_fibre_from_degenerate(-1.6, -1.4)), 'f2f-dot', 'back']
fibre_x1   = [compose(pr, projected_fibre_from_degenerate(-1.4,  1.4)), 'f2f', 'back']
fibre_x1_e = [compose(pr, projected_fibre_from_degenerate( 1.4,  1.6)), 'f2f-dot', 'back']
cmds = generate_raw_draw_2d(0.5, [fibre_x2x3, fibre_x1, fibre_x1_b, fibre_x1_e])
box  = generate_raw_box_2d([1.5, 1.5, 1.5], pr, 'box')
write_items('generated/two-fibres-box.tikz', cons(box, cmds))

latm   = generate_raw_latitude_2d(0.0, pr, 'dashed')
points = [[[-pi * 0.5, 0.0], 'f1c'], [[pi * 0.5, 0.0], 'f2c']]
cmds   = generate_raw_points_2d(pr, points, '1.3pt')
write_items('generated/two-fibres-sphere.tikz', cons(latm, cmds))
\end{python}
\begin{center}
\tikzexternalenable
\tikzsetnextfilename{two-fibres}
\begin{tikzpicture}
\tikzstyle{back} = [white, line width = 3pt];
\tikzstyle{f1f} = [f1c, line width = 1pt];
\tikzstyle{f2f} = [f2c, line width = 1pt];
\tikzstyle{f2f-dot} = [f2f, dotted, line cap = round];
\tikzstyle{box} = [line width = 0.47pt];
\coordinate (D) at (-3, 0);
\coordinate (C) at ( 3, 0);

\begin{scope}[shift = {(D)}, scale = 1.2, local bounding box = {domain}]
\input{generated/two-fibres-box.tikz}
\end{scope}

% Draw S^2.
\begin{scope}[shift = {(C)}, scale = 1.5, local bounding box = {codomain}]
\draw[line width = 0.47pt] (0, 0) circle (1);
\input{generated/two-fibres-sphere.tikz}
\end{scope}

% Draw an arrow from the plot to S^2 with tikzcd arrow style.
\begin{scope}[commutative diagrams/every diagram]
\path let \p1 = (domain.east) in node (from) at (\x1 + 1em, 0) {};
\path let \p1 = (codomain.west) in node (to) at (\x1 - 1em, 0) {};
\path[commutative diagrams/.cd, every arrow, every label]
  (from) edge [commutative diagrams/maps to] node {$h \circ \pi^{-1}$} (to);
\end{scope}
\end{tikzpicture}
\end{center}
\end{figure}

\section{Linking}
\label{sec:linking}
The fibres of the Hopf map — circles, as shown in the previous section — possess an interesting property:
they are all linked with every other fibre.
In this section we will give a formal definition of linking,
and prove linkedness of the fibres.
In section~\ref{sec:linked-and-knotted-fields} we will explore a physical application of linking.

Linking is a property of a pair of closed curves
that is not intrinsic to the curves as topological spaces themselves,
but rather to their embedding in a surrounding space.
Considering this, it makes sense to look at the complement of the curves.
By studying the fundamental group of the complement,
we can tell different situations apart.
For instance,
the fundamental group of the complement of two linked circles in $\R^3$
is the free abelian group on two generators,
whereas the fundamental group of the complement of two unlinked circles
is the free \emph{nonabelian} group on two generators.
(See \parencite[p.~46]{hatcher2002}.
Incidentally, Hatcher introduces linking
as one of the main motivations for studying the fundamental group.)

\definition
Let $X$ be a topological space.
An \emph{$n$-link} in $X$ is an ordered collection of $n$ continuous maps $\sigma_i : S^1 \to X$,
such that the images of $\sigma_1, \ldots, \sigma_n$ are disjoint.
The link is called \emph{proper} if every $\sigma_i$ is a homeomorphism onto its image.

This definition is based on \parencite{milnor1954}.
Because in a proper link every $\sigma_i$ is a homeomorphism onto its image,
the components of the link do not self-intersect.
Because the images are disjoint,
they do not intersect eachother.

\definition
Two $n$-links $(\sigma_1, \ldots, \sigma_n)$ and $(\tau_1, \ldots, \tau_n)$
are said to be \emph{homotopic}
if there exist homotopies $H_i : [0, 1] \times S^1 \to X$ from $\sigma_i$ to $\tau_i$,
such that for all $t \in [0, 1]$,
the images of $H_i(t, {}\cdot{})$ are disjoint.
The links are said to be \emph{properly homotopic} if
for all $t \in (0, 1)$ the maps $H_i(t, {}\cdot{})$ are homeomorphisms onto their images.
Homotopy and proper homotopy define two equivalence relations on the set of $n$-links.
We call a proper $n$-link \emph{trivial}
if it is properly homotopic to an $n$-link of $n$ distinct constant functions.

A homotopy between links captures the idea of links being “the same”.
Homotopy enables us to tell apart many different types of links,
but there is one caveat:
a homotopy from one link into another might have self-intersecting components
at some point in time.
For instance, the Whitehead link is homotopic to two unlinked circles,
but it can only be unlinked if the components are allowed to self-intersect.
\marginfigure{
\tikzexternalenable
\tikzsetnextfilename{whitehead-link}
\begin{center}
\hspace{-1em}
\begin{tikzpicture}[baseline={(C.base)}]
\tikzstyle{back} = [line width = 4.94pt, black]
\tikzstyle{knot} = [line width = 4.0pt, white]
\setlength{\kr}{2em}
\coordinate (A) at ( 45 : 2\kr);
\coordinate (B) at (-45 : 2\kr);
\coordinate (C) at ($ (A) !.5! (B) $); % Half-way between (A) and (B).

\useasboundingbox let \p1 = (A), \p2 = (B) in
            (\x1 - \kr - 2.5pt, \y1 + \kr + 2.5pt) rectangle
            (\x2 + \kr + 2.5pt, \y2 - \kr - 2.5pt);

\draw[back] (C) circle (\kr);
\draw[knot] (C) circle (\kr);

\draw[back] ($(B) + (270 : \kr)$) arc (270 : 135 : \kr) -- (C) --
            ($(A) + (-45 : \kr)$) arc (-45 :  90 : \kr);
\draw[knot] ($(B) + (270 : \kr)$) arc (270 : 135 : \kr) -- (C) --
            ($(A) + (-45 : \kr)$) arc (-45 :  90 : \kr);

\draw[back] ($(A) + (89 : \kr)$) arc (89 : 225 : \kr) -- (C) --
            ($(B) + (45 : \kr)$) arc (45 : -91 : \kr);
\draw[knot] ($(A) + (88 : \kr)$) arc (88 : 225 : \kr) -- (C) --
            ($(B) + (45 : \kr)$) arc (45 : -92 : \kr);

\draw[back] ($(C) + (120 : \kr)$) arc (120 : 150 : \kr);
\draw[knot] ($(C) + (119 : \kr)$) arc (119 : 151 : \kr);

\draw[back] ($(C) + (-30 : \kr)$) arc (-30 : -60 : \kr);
\draw[knot] ($(C) + (-29 : \kr)$) arc (-29 : -61 : \kr);
\end{tikzpicture}
\end{center}
\vspace{1em}
\caption{The Whitehead link.}}
With the notion of proper homotopy we can also differentiate between
the Whitehead link and and unlinked circles:
the unlinked circles are trivial,
but the Whitehead link is not.
These types of links are beyond the scope of this thesis though;
for the fibres of the Hopf map
the notion of homotopy will be sufficient.
To determine whether two closed curves are linked,
we will examine the fundamental group of the complement of \emph{one} curve.
The other curve then determines an element of the fundamental group.
If the fundamental group happens to be $\Z$,
we can quantify linking with an integer.

\definition
Let $(\sigma_1, \sigma_2)$ be a proper two-link in a topological space $X$.
Suppose that $G_1 = \pi_1(X \setminus \im \sigma_1,\, \sigma_2(0)) \cong \Z$.
Then $[\sigma_2]$ is an element of $G_1$,
so under an isomorphism $G_1 \to \Z$ it maps to an integer $n$.
Its absolute value $|n|$ is independent of the choice of isomorphism.
This $|n|$ is the \emph{linking number} of $\sigma_2$ with $\sigma_1$.

This definition of linking number is not symmetric with regard to $\sigma_1$ and $\sigma_2$:
we require only the complement of $\sigma_1$ to have a fundamental group isomorphic to $\Z$.
Even in $\R^3\!$, the fundamental group of such a complement can be quite surprising.
For example, the fundamental group of the complement of an $(m, n)$ torus knot
is shown in \parencite[p.~47]{hatcher2002} to be the quotient group
of the free group with generators $a$ and $b$,
where $a^m$ and $b^n$ are identified.
\marginfigure{
\tikzexternalenable
\tikzsetnextfilename{trefoil-knot}
\begin{center}
\hspace{0.5em}
\begin{tikzpicture}[baseline={(C.base)}]
\tikzstyle{back} = [line width = 4.94pt, black]
\tikzstyle{knot} = [line width = 4.0pt, white]
\setlength{\kr}{2em}
\coordinate (A) at ( 90 : 0.7\kr);
\coordinate (B) at (210 : 0.7\kr);
\coordinate (C) at (330 : 0.7\kr);
\coordinate (Z) at ($(B) !.5! (C) + (90 : \kr)$);

\useasboundingbox let \p1 = (A), \p2 = (B), \p3 = (C) in
            (\x2 - \kr - 2.5pt, \y1 + \kr + 2.5pt) rectangle
            (\x3 + \kr + 2.5pt, \y3 - \kr - 2.5pt);

\draw[back] ($(C) + (330 : \kr)$) arc (-30 :  90 : \kr) -- (Z);
\draw[knot] ($(C) + (330 : \kr)$) arc (-30 :  90 : \kr) -- (Z);
\draw[back] ($(B) + (210 : \kr)$) arc (210 : 330 : \kr) --
            ($(A) + (330 : \kr)$) arc (-30 :  90 : \kr);
\draw[knot] ($(B) + (210 : \kr)$) arc (210 : 330 : \kr) --
            ($(A) + (330 : \kr)$) arc (-30 :  90 : \kr);
\draw[back] ($(A) + ( 89 : \kr)$) arc ( 89 : 210 : \kr) --
            ($(C) + (210 : \kr)$) arc (210 : 331 : \kr);
\draw[knot] ($(A) + ( 88 : \kr)$) arc ( 88 : 210 : \kr) --
            ($(C) + (210 : \kr)$) arc (210 : 332 : \kr);
\draw[back] (Z) + (0.1pt, 0) -- ($(B) + ( 90 : \kr)$) arc ( 90 : 211 : \kr);
\draw[knot] (Z) + (0.5pt, 0) -- ($(B) + ( 90 : \kr)$) arc ( 90 : 212 : \kr);
\end{tikzpicture}
\end{center}
\vspace{1em}
\caption{A 2,3 torus knot, also called a \emph{trefoil knot}.}}
This means that the linking number of a nontrivial torus knot with a circle
is well-defined,
but the linking number of the circle with the knot is not.
This problem can be alleviated by considering the first homology group
instead of the fundamental group,
an approach that is taken in \parencite[p.~132]{rolfsen2003}.
Rolfsen also relates the linking number as defined here to other definitions,
such as the \emph{Gauss linking integral}.
In the remainder of this section,
we will only consider curves of which the fundamental group of the complement is isomorphic to $\Z$.
For curves in $\R^3$ or $S^3\!$,
it is shown in theorem~6 of \parencite[p.~135]{rolfsen2003}
that the linking number
does not depend on the order of $\sigma_1$ and $\sigma_2$,
nor on their orientation.
This means that the we can quantify the linking of $\set{\im \sigma_1, \im \sigma_2}$
with a unique nonnegative integer.
A nonzero linking number implies that two curves are linked,
but the converse does not hold:
the Whitehead link has linking number zero,
but it is not trivial.
In any case,
the linking number suffices to show that the fibres of the Hopf map are linked.
Before we prove the general case
we will demonstrate linkedness of two particular fibres.
By using the action of $\GLC$
this proof can be extended to the general case.

As shown in section~\ref{sec:fibres},
the fibres of the Hopf map above $(1 : 0)$ and $(0 : 1)$ may be parametrised as
\[ \sigma_1: [0, 1] \longto S^3, \quad t \longmapsto (e^{2\pi i t}, 0)
   \quad \textup{and} \quad
   \sigma_2: [0, 1] \longto S^3, \quad t \longmapsto (0, e^{2\pi i t}) \]
In corollary~\ref{cor:circles-and-axis} we saw that under stereographic projection,
$\sigma_1$ maps to the $x_1$ axis
and $\sigma_2$ maps to the unit circle in the $x_2 x_3$-plane.

\proposition
Let $\sigma_1$ and $\sigma_2$ be as introduced above.
Then $\sigma_2$ is linked once with $\sigma_1$ in $S^3\!$.

\proof
Restricted to $S^3 \setminus \im \sigma_1$,
the stereographic projection $\pi : S^3 \setminus\set{p} \to \R^3$
is a homeomorphism onto $\R^3 \setminus \pi(\im \sigma_1)$,
because the projection point $p = (1, 0)$ lies on the image of $\sigma_1$.
Therefore, it induces an isomorphism
$\pi_1(S^3 \setminus \im \sigma_1, \, \sigma_2(0)) \to \pi_1(\R^3 \setminus \pi(\im \sigma_1), \, \pi(\sigma_2(0)))$
on fundamental groups.
As we saw before, $\pi(\im \sigma_1)$ is the $x_1$-axis in $\R^3\!$,
so the space $\R^3 \setminus \pi(\im \sigma_1)$ deformation retracts onto $\R^2$ minus the origin
by projecting on the $x_2 x_3$-plane.
This induces an isomorphism
$\pi_1(S^3 \setminus \im \sigma_1, \, \sigma_2(0)) \to \pi_1(\R^2 \setminus \set{0},\, (1, 0))$.
(See for example proposition~1.17 of \parencite[p.~31]{hatcher2002}.)
Because the image of $\pi \circ \sigma_2$ lies in the $x_2x_3$ plane,
$[\pi \circ \sigma_2]$ is an element of $\pi_1(\R^2 \setminus \set{0},\, (1, 0))$.
This fundamental group is of course isomorphic to $\Z$,
and $\pi \circ \sigma_2$ is a curve that goes around the origin once,
so it is a generator of the fundamental group.
It follows that $\sigma_2$ is linked once with $\sigma_1$.
\qed

To demonstrate that any two fibres are linked,
we will improve upon the result of proposition~\ref{prop:glc-acts-transitively},
which stated that $\GLC$ acts transitively on $\PC$.
In fact, the stabiliser $\GLC_p$ of a point $p \in \PC$
still acts transitively on $\PC \setminus \set{p}$.

\proposition[prop:glc-acts-doubly-transitive]
Let $(z_1 : z_2)$ and $(\nu_1 : \nu_2) \in \PC$ be distinct points.
Then there exists a $g \in \GLC$,
such that $g \cdot (1 : 0) = (z_1 : z_2)$ and $g \cdot (0 : 1) = (\nu_1 : \nu_2)$.

\proof
Consider the matrix
\[ g = \begin{pmatrix}z_1 & \nu_1 \\ z_2 & \nu_2 \end{pmatrix} \]
The columns of this matrix are linearly independent by assumption,
so its determinant is nonzero.
It follows that $g \in \GLC$,
and clearly $g \cdot (1 : 0) = (z_1 : z_2)$
and $g \cdot (0 : 1) = (\nu_1 : \nu_2)$.
\qed

\corollary
Any two fibres of the Hopf map are linked in $S^3$:
proposition~\ref{prop:glc-acts-doubly-transitive} tells us that the situation
of any two fibres can be transformed into the situation of $(1 : 0)$ and $(0 : 1)$
by a homeomorphism,
and the linking number is invariant under such a homeomorphism.

\begin{figure}
\margincaption[0.5em]{\label{fig:more-fibres}
Linked fibres of the Hopf map visualised through stereographic projection.
Fibres above points near $ i \in S^2$ (the north pole) are circles with a large radius in $\R^3\!$,
close to the $x_1$-axis (truncated here).
Fibres above points near $-i \in S^2$ (the south pole) are circles close to the unit circle in the $x_2 x_3$-plane.}
\begin{python}
from hopf import *
from math import pi
interval5 = interval_open(5)
interval3 = interval_open(3)
points_a  = [[[-pi * 0.3, (t * 0.7 + 0.30) * pi], 'a{0}'.format(i)] for (i, t) in enumerate(interval5)]
points_b  = [[[-pi * 0.1, (t * 0.4 + 0.50) * pi], 'b{0}'.format(i)] for (i, t) in enumerate(interval5)]
points_c  = [[[ pi * 0.3, (t * 0.5 + 1.50) * pi], 'c{0}'.format(i)] for (i, t) in enumerate(interval3)]
points_ab = points_a + points_b
points    = points_ab + points_c
pr        = orthographic_projection(pi * 0.42, pi * -0.1)
def limited(x):
    f = compose(pr, projected_fibre_from_spherical(x))
    return lambda t: f((0.910 + t * (1.590 - 0.921)) % 1) # Tweaked to the box boundaries.
fibres_ab = [[compose(pr, projected_fibre_from_spherical(x)), 'front, ' + st, 'back'] for [x, st] in points_ab]
fibres_c  = [[limited(x), 'front, ' + st, 'back'] for [x, st] in points_c]
fibres    = fibres_ab + fibres_c
cmds      = generate_raw_draw_2d(0.1, fibres)
box       = generate_raw_box_2d([1.636, 1.636, 1.636], pr, 'box')
write_items('generated/more-fibres-box.tikz', cons(box, cmds))

lat_a   = generate_raw_latitude_2d(-pi * 0.3, pr, 'dashed')
lat_b   = generate_raw_latitude_2d(-pi * 0.1, pr, 'dashed')
lat_c   = generate_raw_latitude_2d( pi * 0.3, pr, 'dashed')
cmds    = generate_raw_points_2d(pr, points, '1.3pt')
write_items('generated/more-fibres-sphere.tikz', cat([lat_a, lat_b, lat_c], cmds))

colours_a = generate_colours('a', [[  0.0 + h * 0.4,  0.6, 0.6] for h in interval5])
colours_b = generate_colours('b', [[  0.5 + h * 0.4,  0.6, 0.6] for h in interval5])
colours_c = generate_colours('c', [[-0.05 + h * 0.27, 0.9, 0.9] for h in interval3])
colours   = cat(cat(colours_a, colours_b), colours_c)
write_items('generated/more-fibres-colours.tikz', colours)
\end{python}
\begin{center}
\tikzexternalenable
\tikzsetnextfilename{more-fibres}
\begin{tikzpicture}
\input{generated/more-fibres-colours.tikz}
\tikzstyle{back}  = [line width = 3pt, white];
\tikzstyle{front} = [line width = 1pt];
\tikzstyle{box} = [line width = 0.47pt];
\coordinate (D) at (-3, 0);
\coordinate (C) at ( 3, 0);

\begin{scope}[shift = {(D)}, scale = 1.1, local bounding box = {domain}]
\input{generated/more-fibres-box.tikz}
\end{scope}

% Draw S^2.
\begin{scope}[shift = {(C)}, scale = 1.5, local bounding box = {codomain}]
\draw[line width = 0.47pt] (0, 0) circle (1);
\input{generated/more-fibres-sphere.tikz}
\end{scope}

% Draw an arrow from the plot to S^2 with tikzcd arrow style.
\begin{scope}[commutative diagrams/every diagram]
\path let \p1 = (domain.east) in node (from) at (\x1 + 1em, 0) {};
\path let \p1 = (codomain.west) in node (to) at (\x1 - 1em, 0) {};
\path[commutative diagrams/.cd, every arrow, every label]
  (from) edge [commutative diagrams/maps to] node {$h \circ \pi^{-1}$} (to);
\end{scope}
\end{tikzpicture}
\end{center}
\end{figure}

Because the stereographic projection is a homeomorphism,
the projection of any two fibres in $S^3$ that do not pass through the projection point
will be a set of two linked circles in $\R^3\!$.
Even if one of the fibres passes through the projection point
(and thus projects to the $x_1$-axis),
there is a sense of linkedness in $\R^3$:
the fibre that does not pass through the projection point
will project to a circle around the $x_1$-axis.
A few of the fibres have been visualised in figure~\ref{fig:more-fibres}.

% TODO
% \theorem
% Let $(\sigma_1, \sigma_2)$ and $(\tau_1, \tau_2)$ be homotopic proper two-links in $\R^3\!$,
% where $\sigma_1$ is linked $n$ times with $\sigma_2$
% and $\tau_1$ is linked $m$ times with $\tau_2$,
% where $m, n$ are nonnegative integers.
% Then $m = n$.
%
% \proof

\subsection*{Informal summary}
In this section we used topology to quantify linkedness.
An \emph{$n$-link} is a collection of non-intersecting closed curves,
and for a \emph{proper $n$-link} the curves cannot be self-intersecting either.
If two links can be defomed into one another by bending and twisting but not intersecting,
the links are called \emph{homotopic}.
With homotopy we allow the curves to self-intersect in the process,
but for a \emph{proper homotopy} even this is disallowed.
If we want to know whether e.g. a collection of rubber bands can be unlinked,
we must ask whether the corresponding link is \emph{trivial}.
If it is, it is possible to separate all of the bands.
Because homotopy does not allow us to quantify linkedness,
we turn to another quantity: the \emph{linking number}.
The linking number of two closed curves $\sigma_1$ and $\sigma_2$
counts how many times $\sigma_2$ winds around $\sigma_1$,
a concept that can be made precise by using the fundamental group.
Finally, we showed that any two fibres of the Hopf map have linking number one in $S^3\!$.
Using stereographic projection,
we can see that the fibres are linked in $\R^3$ as well.
